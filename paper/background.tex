\section{Background}
\label{sec:background}

In this section we cover the background of Namecoin. We begin by looking at the history of the cryptocurrency and then explain why a block chain can be used for our definition of a namespace. We then proceed to discuss the technical design of Namecoin and the mechanisms used in the design. We conclude this section with a discussion about the applications of Namecoin.

\subsection{History}

Namecoin is an alternative cryptocurrency modeled after Bitcoin\cite{nakamoto2008bitcoin}. Furthermore, it is the first altcoin in the sense that it was the first to create its own block chain, separate from Bitcoin's.  Namecoin shares many similarities with Bitcoin, including the same method for proof-of-work, the same coin cap, the same block creation time, and all of the same transaction operations (with a few additions). Namecoin was inspired after discussions about a BitDNS\cite{bitdns} protocol using a block chain to manage a domain name lookup service. The idea behind this was that a central authority managing domain names, such as ICANN, requires too much trust in a single entity. Having a decentralized name system described by BitDNS provides resistance against a single point of failure. The first Namecoin block was mined in April 2011, and as of the writing of this paper, over 215,000 total blocks have been mined in the Namecoin system. Because of its similarities with Bitcoin, Namecoin was able to be merge-mined and has been merge-mined with Bitcoin since October 8, 2011. 

\subsection{Description of the Block Chain}
\cite{bonneau2014decentralizing}
Namecoin is minted and maintained by a decentralized peer-to-peer network. Namecoin transactions require the digital signature of the account holder to prevent theft, and every transaction is published in an append-only hash chain, called the block chain. The block chain can be extended with new transactions by any participant, and such participants (informally known as miners) may obtain newly minted Namecoin currency (NMC) and/or tips from the transactions for performing this function. Extensions to the block chain require a proof-of-work that rate-limits the process (to approximately an extension every ten minutes) which enables a steady inflation rate, ample competition among participants to extend the block chain, and adequate time to obtain and verify the history of the block chain for new participants. Informally, the proof-of-work protocol in Namecoin is intended to maintain the following two essential properties about the block chain:

\begin{itemize}

  \item Every party eventually agrees on the order and correctness of transactions in the block chain.
  \item Any party can publish a transaction (for a fee), which will then be verified and, if valid, included in the block chain within a small bounded delay

\end{itemize}

A use case for a block chain is a tamper-evident log. That is, it can be used as a data structure that stores a bunch of data, and allows users to append data onto the end of the log. Each new block has a hash of the previous block, so if somebody alters data that is earlier in the log, it will be detected. If an adversary wants to tamper with data anywhere in this entire chain, in order to keep the story consistent, he's going to have to tamper with the hash pointers all the up to and including the current block. Thus it emerges, that by just remembering this single hash pointer of the head of the chain, users essentially remember a tamper-evident hash of the entire list, all the way back to the genesis block. Namecoin, and all block chain based cryptocurrencies, use this tamper-evident log in order to record all the transactions between individuals. 


{\bf Block chain security.}
The value and stability of a cryptocurency are directly related to the amount of proof-of-work involved in the calculations of the blocks because this work is what keeps the data distributed and secure in the tamper-evident block chain. For both Bitcoin and Namecoin, the proof-of-work is shown by calculating the hash of a new block and a random nonce over and over until the calculated hash has a certain number of leading zeros. The number of leading zeros required by the hash is referred to as the difficulty threshold. Namecoin enjoys a very high difficulty threshold for the proof-of-work because it is similar to Bitcoin and supports merge mining with Bitcoin. This allows miners who are mining Bitcoin to also mine Namecoin at the same time with no extra work for the miner. Essentially, this is because the miner is using their computational power to solve a cryptographic puzzle that satisfies the proof-of-work for both block chains at the same time. This is advantageous for the miners because they are rewarded with coins from both systems, and helps Namecoin because it gives the Namecoin network a vastly increased amount of hash power over what it would have if it did not support merged mining. The high hash power and difficulty of the cryptographic challenge on the Namecoin network increases the stability of the system, providing resilience to a 51\% attack and other threatening behaviors from malicious miners.

\subsection{Technical Details of Names}

In this section and the next, we make an effort to differentiate between the technical details of the Namecoin name/value implementation and the specific mechanism design choices in Namecoin. The feature that separates Namecoin from Bitcoin is that Namecoin is a namespace, and can be used to register name/value pairs that can be stored in the block chain and traded amongst individuals. This registration is done using the three script operations exclusive to Namecoin: {\tt NAME\_NEW}, {\tt NAME\_FIRSTUPDATE}, and {\tt NAME\_UPDATE}. In order to understand the registration process, we think it is helpful to walk through the registration process of {\tt name}, roughly following Figure \ref{fig:registration}. 

{\bf NAME\_NEW.}
To start, the user will need to select a coin to be crafted into a token (or special coin) that represents a name and whose value can be changed by whoever possess the token. The next step to register a name is to make a transaction that uses the {\tt NAME\_NEW} script operation in a transaction sending the token from one of their addresses to another. Using {\tt NAME\_NEW}, a user can indicate an interest in {\tt name} for a name/value pair by posting a hash commitment of the desired {\tt name} in the scriptPubKey of the transaction. The {\tt NAME\_NEW} operation acts as a signal in the block chain for name parsers to indicate that the next part of the scriptPubKey will be the hash commitment to a name. The protocol then places an {\tt OP\_2DROP} on the stack to remove all of the name information put on the stack with the {\tt NAME\_NEW} operation so that the rest of the locking portion of the scriptPubKey can function just as it does in Bitcoin. 

\begin{figure*}
  \centering
  \includegraphics[width=0.95\textwidth]{figures/registration.png}
  \caption{Namecoin Name Registration Protocol}
  \label{fig:registration}
\end{figure*}

{\bf NAME\_FIRSTUPDATE.}
After doing this, and waiting for 12 or more blocks on top of the one containing the {\tt NAME\_NEW} transaction, the same user can use the output of the {\tt NAME\_NEW} transaction as the input for the {\tt NAME\_FIRSTUPDATE} transaction. Once completed, this will associate the chosen {\tt name} with {\tt value} selected by the user. Similar to {\tt NAME\_NEW}, {\tt NAME\_FIRSTUPDATE} allows data to be posted in the block chain as part of the scriptPubKey of a special transaction. 
To create a {\tt NAME\_NEW} transaction, a user will select, as input, the output of the {\tt NAME\_NEW transaction}. They will then use another address they control as the output for the transaction. The scriptPubKey of this transaction will contain a {\tt NAME\_FIRSTUPDATE}, the {\tt name} desired, the random nonce used in the {\tt NAME\_NEW} hash commitment, the first {\tt value} for the name to take and some {\tt OP\_DROP}s to clear these items from the stack so that they don't interfere with the rest of the locking script, which comes after these extra components. In order for this transaction to be valid, a miner will verify that the {\tt name} and the provided nonce do, in fact, hash to the commitment in the appropriate {\tt NAME\_NEW} transaction. The output of this transaction now contains the token representing the name/value pair for {\tt name} and {\tt value}, and whoever can unlock and spend the output can utilize the final new operation, {\tt NAME\_UPDATE}.

{\bf NAME\_UPDATE.}
The third and final new operation in Namecoin is the {\tt NAME\_UPDATE} operation. Again, this operation's arguments (the {\tt name} and {\tt newValue}) are stored in the scriptPubKey of a special transaction. This transaction must have as input a {\tt NAME\_FIRSTUPDATE} or {\tt NAME\_UPDATE} output with the same {\tt name}. This operation has three primary uses: updating, renewing and trading a name. If the user wants to change the {\tt value} associated with a given {\tt name}, they will update {\tt name} with this operation, providing a {\tt newValue}. If names can expire, as they do in Namecoin, then this operation can also be used to renew a name by providing a {\tt newValue} that is the same as the old {\tt value}. In either of these cases, the user will use an address they control as an output of the transaction. The final reason to make a {\tt NAME\_UPDATE} transaction is to trade the special coin to another user. In this case, the user will put, as an output, one of the other user's addresses instead of their own. Once the transaction resolves, the other user will have control over the special coin and can change the value to whatever they deem fit. Because the ownership of {\tt name} is associated with the ownership of the special coin, if the buyer is paying for the name with Namecoins, the exchange between the payment and the name can be atomic (meaning they happen in the same transaction and either are only valid if the other is as well). 

\subsection{Mechanism Design}

In this subsection we continue discussion about the details of Namecoin. As opposed to the last section, however, this section focuses on the specifics of the mechanism design in Namecoin. 

{\bf Fees.}
Namecoin has been implemented with various fees and protocols to incentivize the behaviours of the users. The special token used in the {\tt NAME\_NEW} transaction has a value of 0.01 NMC. This coin will be not be spendable like other Namecoins while it has a name attached to it, but if the name expires, then the coin will resume function as usual. For all of the transactions, {\tt NAME\_NEW}, {\tt NAME\_FIRSTUPDATE}, and {\tt NAME\_UPDATE}, the default behaviour is to have the user tip the miner. The current expected tip, which is programmed into the Namecoin client, is 0.005 NMC on each transaction. Historically, Namecoin also had a network fee attached to the {\tt NAME\_FIRSTUPDATE} transaction. The network fee is different from the transaction fee; the transaction fee is paid to the miners, whereas the network fee was destroyed (with an {\tt OP\_RETURN}) when a {\tt NAME\_FIRSTUPDATE} transaction was confirmed. The network fee varied over time-- it started at 50 NMC at the genesis block, but decreased by a factor of 2 every 8192 blocks (which is approximately 2 months). The purpose of the network fee was to have a large initial cost to claiming names to deter users from quickly claiming all the desirable names, but then decay off so that eventually the cost of registering a name becomes negligible. As of block 85585, the network fee became small enough that it rounds to 0 and is no longer added onto the transaction. The current implementation of Namecoin has no fees other than the transaction fees.

{\bf Expiration.}
Namecoin has an expiration time for names. Originally, the time period for a name to expire was 12,000 blocks, but by March 2012, the expiration period was increased to 36,000 blocks (which comes out to about 250 days). If a particular {\tt name} hasn't been mentioned in a {\tt NAME\_FIRSTUPDATE} or {\tt NAME\_UPDATE} in 36,000 blocks, the name becomes available again for any user to claim with {\tt NAME\_NEW} and {\tt NAME\_FIRSTUPDATE}. Similarly, a {\tt NAME\_UPDATE} must cite a {\tt NAME\_FIRSTUPDATE} or {\tt NAME\_UPDATE} that is less than 36,000 blocks old as input.

{\bf Changing the Protocol.}
Although it is difficult to change the protocol, it is not impossible and has already been done in Namecoin. Both the addition of merge mining with Bitcoin, and the increase in expiration length were not initially in the design of Namecoin. Both of these changes to the system required a hard fork in the block chain. In order to make these changes, the Namecoin community decided on arbitrary blocks at which point the new protocol would be enforced. For example, up to block 19,199 blocks that were merge mined with Bitcoin were not allowed into the Namceoin block chain, but starting on block 19,200 they were. 
 
\subsection{Applications}

There are many different subspaces in the Namecoin namespace, and the different subspaces have different applications. When claiming a name, a user prepends the name with a subspace ID and a slash. Namecoin was created to be very general so that it would be useful for any application that would benefit from an online name/value store. While {\tt d/} has the most registered names, there are many used subspaces in Namecoin.

{\bf .bit Domains.}
The vision for Namecoin was to use one of these subspaces for DNS lookup in the .bit TLD. Explicitly, Namecoin names associated with .bit domains are prepended with the subspace ID {\tt d/}. If a user wanted the domain awesome.bit, they would claim the name {\tt d/awesome}. The owner of awesome.bit would then set the value to their server address in a way that would be understood by .bit compliant DNS servers as described in the .bit specification\cite{bitdnsspec}.

Most major web servers, such as Apache, ngnix, and lighthttpd will accept connections through .bit domains with minor modifications to their per-site configuration files.

{\bf OneName.}
Analyzing the names on the blockchain reveals that the online identity service, OneName, has a very large user base. After limiting the number of name/value pairs to only those with unique, well formed entries (using the strategies listed in section~\ref{sec:methods}) OneName has roughly 20,000 pairs, which dwarfs the 1000 or so in the d/ namespace. This suggests that Namecoin can be successfully used for applications beyond DNS. The idea behind OneName is that a user can have a name/value pair in the blockchain that associates said name with different online identities such as an email, GitHub, or Twitter. A OneName user can then confirm ownership of accounts on any of these services by referencing their OneName name through some messaging channel in each respective system. For example, if Alice wants to tie her twitter username to her OneName username, she must tweet the message ``Verifying that +Alice is my openname (my Bitcoin username). https://onename.com/Alice''. This is similar to the verification scheme Keybase uses.

OneName completely abstracts away the involvement of Namecoin for its users. In order to make a OneName identity, a user only has to visit OneName's website to create an account. OneName takes the information given to it and puts it into a name/value pair and posts that into the Namecoin blockchain. The user doesn't need to know anything about Namecoin, have Namecoins, or even know that Namecoin is working behind the scenes in order to complete the process. One reason this model works for OneName is because name/value pairs have become quite cheap. Registering a name only costs 0.01 NMC for the token and 0.01 NMC to cover two transactions fees. It then only costs 0.005 NMC every 250 days or so to update the name/values to keep them fresh. Given the current exchange rate of Namecoin to dollars, this represents a cost of approximately one cent for OneName to post one of its user's information into the blockchain. Even with a user base as large as 20,000, OneName's expenses to cover everyone's fees total to only a few hundred dollars~\cite{exchangerate}. OneName has made the assumption that picking up the tab on the fees is well worth removing the effort of trying to use Namecoin for thie users, and it seems that this approach is successful.
